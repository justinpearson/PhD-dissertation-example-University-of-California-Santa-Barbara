\newif\ifIEEE                          % if ieee conf or journal (as opposed to draft or tech report)
\IEEEfalse 
% \IEEEtrue      % DONT FORGET TO TOGGLE \techreptrue/false TOO! AND
               % ALSO ONECOL:

\newif\ifIEEEonecol
% \IEEEonecoltrue    % double-spaced etc. Just used for \documentclass,
                   % to get the one column format that ieee requires
                   % as supplementary material.
\IEEEonecolfalse  

\newif\iftechrep    % if technical report (proof of lemm3,5,6
                     % included in tech rep, omitted from May 2016 TAC
                     % journal bc of space.)
% \techrepfalse
\techreptrue

%%%%%%%%%%%%%%%%%%%%%%%%%%%%%%%%%%%%%%%%%%%%%%%%%%%%%%%%%%%%%%%




\chapter{Control with Minimum Energy Per Symbol}
\label{chap:mineng}

Parts of this chapter come from \cite{PearsonHespanhaLiberzonMay2017}:

2017 IEEE. Reprinted, with permission, from J. Pearson, J. Hespanha, D. Liberzon. Control with minimal cost-per-symbol encoding and quasi-optimality of event-based encoders. IEEE Trans. on Automat. Contr., 62(5):2286--2301, May 2017.


In this chapter we consider the problem of stabilizing a continuous-time linear
time-invariant process subject to communication constraints. We develop a framework for exploring the notion that the absence of communication nevertheless conveys information, yet it consumes no communication resources. We model the absence of a communication by appending a special ``free'' symbol to the set of symbols offered by the communications channel. Transmitting a normal symbol costs one unit of communications resources, but transmitting the free symbol costs no resources. This yields the notion of an encoder's \emph{\avecost{}} --- essentially the average fraction of \nonfree{} symbols sent by the encoder. We then develop a condition under which a stabilizing encoder with the smallest \avecost{} may be designed.


This chapter is organized as follows. 

...

\section{Problem Statement}
\label{sxn:min-bit-rate}


Consider a stabilizable linear time-invariant process
\begin{align}\label{eq:full-process}
\dot x = A x + B u, \qquad x \in \R^n, u \in \R^m,
\end{align}
for which it is known that $x(0)$ belongs to a known bounded set
$\scr{X}_0 \subset \R^n$. 




\section{Conclusion}


In this chapter, we considered the problem of bounding the state of a continuous-time
linear process under communication constraints. We considered
constraints on both the channel \bitrate{} and the encoding scheme's \avecost{}.
Our main contribution was a necessary and sufficient
condition on the process and constraints for which a bounding
encoder/decoder/controller exists. 

...
